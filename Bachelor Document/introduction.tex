\chapter{Introduction}

The motivation for this bachelor thesis arose from the \emph{AU-SAT Workshop} meeting that took place in late November 2015. The workshop discussed the possibility of the Department of Physics and Astronomy of Aarhus University building a small satellite named the AU-SAT. 

The AU-SAT should make use of new technology to allow for a design much smaller than traditional satellites in space, making the mission affordable for Aarhus University. The satellite would be build in partnership with the Danish small satellite company \emph{GomSpace}.

The idea for a small satellite mission designed by the Department of Physics and Astronomy of Aarhus University, was inspired by two earlier small satellite missions, MOST and BRITE, designed by the Space Flight Laboratory and the Institute for Aerospace Studies at the University of Toronto respectively. These previous missions had shown that science could be preformed with observations made from micro- and nano-satellites.
\\
\\
The AU-SAT mission, however, is not meant to be a copy of the MOST and BRITE missions. AU-SAT will have to be able to preform observations that are not possible from any other small satellite mission. BRITE and MOST are photometric missions, whereas AU-SAT would spectroscopic. The AU-SAT would be a satellite using a spectrograph with a wide wavelength range, and still be able to fit inside a shoe box.
The possible missions for the AU-SAT include,

\begin{itemize}
\item UV spectrum observations of stars.
\item Observations of newly discovered eclipsing binary stars by NASA's TESS mission, allowing the Institute of Physics and Astronomy of Aarhus University to be on the front line for new discoveries.
\item Observations of stellar pulsations, allowing for a better understanding of the interiors of stars. AU-SAT would be the first to use a spectrograph capable of making these observations over such a broad band. 
\item Observe exo-planet candidates discovered by TESS, making discoveries of new exo-planets possible.
\end{itemize}

The conclusion of the AU-SAT workshop was to continue examining the possibility for creating the AU-SAT mission and start testing a possible small spectrograph, for use onboard the AU-SAT, with which this thesis is concerned. 

