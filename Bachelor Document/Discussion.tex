\chapter{Discussion}

The three developed experiments were to simulate the conditions the spectrograph would be exposed to if it were placed in space. 

The first experiment took measurements over a long period of time with the spectrograph being stationary. The results from the first experiment, showed that using a single measurement from the spectrograph meant that the uncertainties of the found wavelengths were of the order of \SI{1}{\nano\meter}. Further analysis of the data showed that binning measurements together and averaging their values, meant that the wavelengths could be identified with uncertainties of the order of \SI{d-3}{\nano\meter}. The use of multiple measurements is a method already used to average the cosmic rays producing additional counts in pixels on the CCD \citep{Mons}. Stacking of measurements to determine the wavelength was therefor used in the following two experiments. 

It was clear from the first experiment that it would be ideal to do further tests of the spectrograph to investigate the cause of the "shoulder" peak in the measurements, shown in fig. \ref{fig: double gauss}. To determine the cause of this, the spectrograph would have to bee taken apart for further examination which was beyond the scope of this thesis. It would be ideal to run the experiments again once the cause of the "shoulder" peak has been identified and removed. 
The final results of the first experiment suggested that the wavelength midrange has the largest stability in determining the wavelengths of the emission lines, corresponding to a smaller uncertainty of the found wavelength as shown in fig. \ref{fig: exp1 usikkerhed peaks}. This was expected as the strongest emission line was located in this range. The order of the uncertainty was also expected to be in the found range as the resolution of the spectrograph was \SI{0.01}{nm}.
\\
\\
For the second experiment the precision of the wavelengths measured for the six peaks was examined under simulation of jitter due to pointing of the satellite. The measurements were binned and an average wavelength was found for each peak. The results of the second experiment found that the wavelengths for the peaks were measured with larger uncertainties than in the first experiment. Opposite the first experiment, the second experiment suggests that the stability of the wavelengths measured for the peaks is smallest in the midrange of the spectrograph. This was not the expected result for the second experiment. It was expected that the uncertainties would be of a larger order due to the movement of the spectrograph, causing the light to hit the slit from different angles. It was, however, also expected that the wavelength midrange would be the must stable, which turned out to not be the case. The reason for the change of the most stable range compared to the first experiment is not known.
This indicates the need for the experiment to be run again to see if the results can be reproduced before concluding on the final results. 
\\
\\
The third experiment examined the shift in measured wavelengths for the six peaks as a function of temperature. For peak 1 to peak 3, which cover the range from \SI{388}{\nano\meter} to \SI{503}{\nano\meter}, the results shown in fig. \ref{fig: exp3 1_6} showed that peaks were shifted to higher wavelengths. The results also indicated that the wavelength shifts for peak 1 to 3 decreased with increasing wavelength.
\\
For peak 4 to 6, which cover the range from \SI{587}{\nano\meter} to \SI{706}{\nano\meter}, the results showed that the peaks were shifted to lower wavelengths. The shift to lower wavelength increased for the peaks going from peak 4 to 6. 
\\
\\
Furthermore from the results it seemed that the shifts for the peaks 4 to 6, were linearly dependent of temperature. For peak 1 to 3 it was not clear that the shifts were linearly dependent on temperature, as the measurements around \SI{30}{\degreeCelsius} were outliers for this to be the case. However, based on the results for peak 4 to 6, the assumption that the shifts for all peaks were linear dependent of temperature was made. The identified wavelength for each peak was then fitted to a linear function of the temperature, so the slope for each fit could be compared. This was done in fig. \ref{fig: exp3 slope coeff.} which suggested that the slopes for the different fits were themselves linear dependent of the wavelength. It was expected that the measured wavelengths were dependent on the temperature, but the fact that the shift as a function of temperature was not the same for all the emission lines was not expected. The cause of the none uniform shift over the wavelength range is expected to lay within the spectrograph, but pinpointing the exact cause is not possible without taking the spectrograph apart and conducting further experiments.
For the results of the third experiment to be verified, the experiment would need to be run several times again and produce similar results before being able to conclude that they are in fact correct. Especially the assumption that the wavelength shifts for all the emission lines are linearly dependent of temperature, would need to be verified by further experiments.

The experiments designed to test the spectrograph have shown that the experiment to be done from this point forward, would include taking it apart to examine what is happening inside during the experiments, especially during the second and third experiment. 


