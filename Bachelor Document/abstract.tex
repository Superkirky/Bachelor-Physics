\chapterstyle{abstract}
\vspace{6cm}
\chapter*{Summary}
%\lipsum[1-3]
\chapterstyle{box}

In a workshop aimed at discussing the possibility of the Department for Physics and Astronomy Aarhus University, launching a nanosatellite space mission, it was decided to test a potential spectrograph to fly on the mission. In this thesis three experiments were designed to simulate the environment the satellite would be subject to in space to test the spectrograph. The stability of measurements from an of-the-shelf spectrograph was examined for stationary conditions and for simulated pointing of the satellite via vibrations. The stationary measurement showed that the spectrograph was able to measure the wavelength for six of the emission lines of Helium, with an uncertainty ranging from \SIrange{0.1d-3}{1.1d-3}{\nano\meter} and for the simulated pointing with the uncertainty in the range \SIrange{0,01}{0,54}{\nano\meter}. Furthermore the temperature variations for a satellite i orbit were simulated to examine the dependency of the temperature for measurements from the spectrograph. It was found the measured wavelength for the Helium emission lines were shifted linearly as a function of temperature, but the shifts over the wavelength range of the spectrograph were not uniform. The range of the shift in wavelength as a function of temperature was \SIrange{-0,5}{0,35}{\nano\meter}. Finally it was concluded that further testing of the spectrograph was needed to verify the results.