\chapter{Conclusion}
In this thesis the USB4000 spectrograph was examined to illuminate the possibilities and limitations for using a compact spectrograph for observations in space onboard a Cubesat. Taking in to consideration the environment satellites are subject to, three experiments were developed and build to test the measurements from the spectrograph, on six of the emission lines of Helium. 
\\
\\
The first experiment examined the stability of stationary measurements over a long period of time. From the first experiment it was found that noise in form of "shoulder" peaks was present at a constant distance from the true peaks in the measured spectra. It was found that the noise from the "shoulder" peaks could be filtered out in the analysis so the wavelengths for the true peaks could be measured with the uncertainties in the range \SIrange{0.1d-3}{1.1d-3}{\nano\meter}.

In the second experiment the stability of measurements when simulating pointing of the satellite, by vibrations was examined. It was measured that with simulated pointing the wavelengths of the lines could be measured with uncertainties in the range \SIrange{0,01}{0,54}{\nano\meter}.

The third experiment tested the effect on the measurements taken when the temperature was varied. It was measured that the wavelength of the peaks were shifted when the temperature was varied for the measurements, but the cause of the shift was not found. The results from the third experiment showed that the shifts of the six peaks were not identical. A theory was suggested that the rate of change of wavelength for the different peaks was linearly dependent of wavelength however it was concluded that further experiments must be conducted with similar results in order to verify this theory.
